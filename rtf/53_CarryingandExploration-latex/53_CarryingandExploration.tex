%&pdfLaTeX
% !TEX encoding = UTF-8 Unicode
\documentclass{article}
\usepackage{ifxetex}
\ifxetex
\usepackage{fontspec}
\setmainfont[Mapping=tex-text]{STIXGeneral}
\else
\usepackage[T1]{fontenc}
\usepackage[utf8]{inputenc}
\fi
\usepackage{textcomp}

\usepackage{array}
\usepackage{amssymb}
\usepackage{fancyhdr}
\renewcommand{\headrulewidth}{0pt}
\renewcommand{\footrulewidth}{0pt}

\begin{document}

This material is Open Game Content, and is licensed for public use under the terms 
of the Open Game License v1.0a.

{\LARGE{}CARRYING, MOVEMENT, \& EXPLORATION}

\vspace{12pt}
\section*{{\LARGE{}CARRYING CAPACITY}}

Encumbrance rules determine how much a character's armor and equipment slow him 
or her down. Encumbrance comes in two parts: encumbrance by armor and encumbrance 
by total weight.

\textbf{Encumbrance by Armor:} A character's armor defines his or her maximum Dexterity 
bonus to AC, armor check penalty, speed, and running speed. Unless your character 
is weak or carrying a lot of gear, that's all you need to know. The extra gear 
your character carries won't slow him or her down any more than the armor already 
does.

If your character is weak or carrying a lot of gear, however, then you'll need 
to calculate encumbrance by weight. Doing so is most important when your character 
is trying to carry some heavy object.

\textbf{Weight:} If you want to determine whether your character's gear is heavy 
enough to slow him or her down more than the armor already does, total the weight 
of all the character's items, including armor, weapons, and gear. Compare this 
total to the character's Strength on Table: Carrying Capacity. Depending on how 
the weight compares to the character's carrying capacity, he or she may be carrying 
a light, medium, or heavy load. Like armor, a character's load affects his or her 
maximum Dexterity bonus to AC, carries a check penalty (which works like an armor 
check penalty), reduces the character's speed, and affects how fast the character 
can run, as shown on Table: Carrying Loads. A medium or heavy load counts as medium 
or heavy armor for the purpose of abilities or skills that are restricted by armor. 
Carrying a light load does not encumber a character.

If your character is wearing armor, use the worse figure (from armor or from load) 
for each category. Do not stack the penalties.

\textbf{Lifting and Dragging:} A character can lift as much as his or her maximum 
load over his or her head.

A character can lift as much as double his or her maximum load off the ground, 
but he or she can only stagger around with it. While overloaded in this way, the 
character loses any Dexterity bonus to AC and can move only 5 feet per round (as 
a full-round action).

A character can generally push or drag along the ground as much as five times his 
or her maximum load. Favorable conditions can double these numbers, and bad circumstances 
can reduce them to one-half or less.

\textbf{Bigger and Smaller Creatures:} The figures on Table: Carrying Capacity 
are for Medium bipedal creatures. A larger bipedal creature can carry more weight 
depending on its size category, as follows: Large x2, Huge x4, Gargantuan x8, Colossal 
x16. A smaller creature can carry less weight depending on its size category, as 
follows: Small x3/4, Tiny x1/2, Diminutive x1/4, Fine x1/8.

Quadrupeds can carry heavier loads than characters can. Instead of the multipliers 
given above, multiply the value corresponding to the creature's Strength score 
from Table: Carrying Capacity by the appropriate modifier, as follows: Fine x1/4, 
Diminutive x1/2, Tiny x3/4, Small x1, Medium x1-1/2, Large x3, Huge x6, Gargantuan 
x12, Colossal x24.

\textbf{Tremendous Strength:} For Strength scores not shown on Table: Carrying 
Capacity, find the Strength score between 20 and 29 that has the same number in 
the ``ones'' digit as the creature's Strength score does and multiply the numbers 
in that for by 4 for every ten points the creature's strength is above the score 
for that row.

\vspace{12pt}
\begin{tabular}{|>{\raggedright}p{67pt}|>{\raggedright}p{59pt}|>{\raggedright}p{64pt}|>{\raggedright}p{59pt}|}
\hline
\multicolumn{4}{|p{251pt}|}{\subsection*{T\textbf{able: Carrying Capacity}}}\tabularnewline
\hline
S\textbf{trength Score} & L\textbf{ight Load} & M\textbf{edium Load} & H\textbf{eavy 
Load}\tabularnewline
\hline
1 & 3 lb. or less & 4-6 lb. & 7-10 lb.\tabularnewline
\hline
2 & 6 lb. or less & 7-13 lb. & 14-20 lb.\tabularnewline
\hline
3 & 10 lb. or less & 11-20 lb. & 21-30 lb.\tabularnewline
\hline
4 & 13 lb. or less & 14-26 lb. & 27-40 lb.\tabularnewline
\hline
5 & 16 lb. or less & 17-33 lb. & 34-50 lb.\tabularnewline
\hline
6 & 20 lb. or less & 21-40 lb. & 41-60 lb.\tabularnewline
\hline
7 & 23 lb. or less & 24-46 lb. & 47-70 lb.\tabularnewline
\hline
8 & 26 lb. or less & 27-53 lb. & 54-80 lb.\tabularnewline
\hline
9 & 30 lb. or less & 31-60 lb. & 61-90 lb.\tabularnewline
\hline
10 & 33 lb. or less & 34-66 lb. & 67-100 lb.\tabularnewline
\hline
11 & 38 lb. or less & 39-76 lb. & 77-115 lb.\tabularnewline
\hline
12 & 43 lb. or less & 44-86 lb. & 87-130 lb.\tabularnewline
\hline
13 & 50 lb. or less & 51-100 lb. & 101-150 lb.\tabularnewline
\hline
14 & 58 lb. or less & 59-116 lb. & 117-175 lb.\tabularnewline
\hline
15 & 66 lb. or less & 67-133 lb. & 134-200 lb.\tabularnewline
\hline
16 & 76 lb. or less & 77-153 lb. & 154-230 lb.\tabularnewline
\hline
17 & 86 lb. or less & 87-173 lb. & 174-260 lb.\tabularnewline
\hline
18 & 100 lb. or less & 101-200 lb. & 201-300 lb.\tabularnewline
\hline
19 & 116 lb. or less & 117-233 lb. & 234-350 lb.\tabularnewline
\hline
20 & 133 lb. or less & 134-266 lb. & 267-400 lb.\tabularnewline
\hline
21 & 153 lb. or less & 154-306 lb. & 307-460 lb.\tabularnewline
\hline
22 & 173 lb. or less & 174-346 lb. & 347-520 lb.\tabularnewline
\hline
23 & 200 lb. or less & 201-400 lb. & 401-600 lb.\tabularnewline
\hline
24 & 233 lb. or less & 234-466 lb. & 467-700 lb.\tabularnewline
\hline
25 & 266 lb. or less & 267-533 lb. & 534-800 lb.\tabularnewline
\hline
26 & 306 lb. or less & 307-613 lb. & 614-920 lb.\tabularnewline
\hline
27 & 346 lb. or less & 347-693 lb. & 694-1,040 lb.\tabularnewline
\hline
28 & 400 lb. or less & 401-800 lb. & 801-1,200 lb.\tabularnewline
\hline
29 & 466 lb. or less & 467-933 lb. & 934-1,400 lb.\tabularnewline
\hline
+10 & x4 & x4 & x4\tabularnewline
\hline
\end{tabular}

\vspace{12pt}
\begin{tabular}{|>{\raggedright}p{37pt}|>{\raggedright}p{42pt}|>{\raggedright}p{65pt}|>{\raggedright}p{32pt}|>{\raggedright}p{32pt}|>{\raggedright}p{22pt}|}
\hline
\multicolumn{6}{|p{232pt}|}{T\textbf{able: Carrying Loads}}\tabularnewline
\hline
  &   &  ------- & \multicolumn{3}{p{86pt}|}{ \textbf{Speed -------}}\tabularnewline
\hline
L\textbf{oad} & M\textbf{ax Dex} & C\textbf{heck Penalty} & (\textbf{30 ft.)} & (\textbf{20 
ft.)} & R\textbf{un}\tabularnewline
\hline
Medium & +3- & 3 & 20 ft. & 15 ft. & x4\tabularnewline
\hline
Heavy & +1- & 6 & 20 ft. & 15 ft. & x3\tabularnewline
\hline
\end{tabular}

\vspace{12pt}
Armor and Encumbrance for Other Base Speeds

The table below provides reduced speed figures for all base speeds from 20 feet 
to 100 feet (in 10-foot increments).

\begin{tabular}{|>{\raggedright}p{46pt}|>{\raggedright}p{64pt}|>{\raggedright}p{46pt}|>{\raggedright}p{64pt}|}
\hline
B\textbf{ase Speed} & R\textbf{educed Speed} & B\textbf{ase Speed} & R\textbf{educed 
Speed}\tabularnewline
\hline
20 ft. & 15 ft. & 70 ft. & 50 ft.\tabularnewline
\hline
30 ft. & 20 ft. & 80 ft. & 55 ft.\tabularnewline
\hline
40 ft. & 30 ft. & 90 ft. & 60 ft.\tabularnewline
\hline
50 ft. & 35 ft. & 100 ft. & 70 ft.\tabularnewline
\hline
60 ft. & 40 ft. &  & \tabularnewline
\hline
\end{tabular}

\vspace{12pt}
{\LARGE{}MOVEMENT}

There are three movement scales, as follows.• 

Tactical, for combat, measured in feet (or squares) per round.• 

Local, for exploring an area, measured in feet per minute.• 

Overland, for getting from place to place, measured in miles per hour or miles 
per day.

\vspace{12pt}
\textbf{Modes of Movement:} While moving at the different movement scales, creatures 
generally walk, hustle, or run.

\textit{Walk: }A walk represents unhurried but purposeful movement at 3 miles per 
hour for an unencumbered human.

\textit{Hustle: }A hustle is a jog at about 6 miles per hour for an unencumbered 
human. A character moving his or her speed twice in a single round, or moving that 
speed in the same round that he or she performs a standard action or another move 
action is hustling when he or she moves.

\textit{Run (x3): }Moving three times speed is a running pace for a character in 
heavy armor. It represents about 9 miles per hour for a human in full plate.

\textit{Run (}x\textit{4): }Moving four times speed is a running pace for a character 
in light, medium, or no armor. It represents about 12 miles per hour for an unencumbered 
human, or 8 miles per hour for a human in chainmail.

\vspace{12pt}
TACTICAL MOVEMENT

Use tactical movement for combat. Characters generally don't walk during combat---they 
hustle or run. A character who moves his or her speed and takes some action is 
hustling for about half the round and doing something else the other half.

\vspace{12pt}
\textbf{Hampered Movement:} Difficult terrain, obstacles, or poor visibility can 
hamper movement. When movement is hampered, each square moved into usually counts 
as two squares, effectively reducing the distance that a character can cover in 
a move. 

If more than one condition applies, multiply together all additional costs that 
apply. (This is a specific exception to the normal rule for doubling) 

In some situations, your movement may be so hampered that you don't have sufficient 
speed even to move 5 feet (1 square). In such a case, you may use a full-round 
action to move 5 feet (1 square) in any direction, even diagonally. Even though 
this looks like a 5-foot step, it's not, and thus it provokes attacks of opportunity 
normally. (You can't take advantage of this rule to move through impassable terrain 
or to move when all movement is prohibited to you.)

You can't run or charge through any square that would hamper your movement.

\vspace{12pt}
LOCAL MOVEMENT

Characters exploring an area use local movement, measured in feet per minute.

\textbf{Walk:} A character can walk without a problem on the local scale.

\textbf{Hustle: }A character can hustle without a problem on the local scale. See 
Overland Movement, below, for movement measured in miles per hour.

\textbf{Run:} A character with a Constitution score of 9 or higher can run for 
a minute without a problem. Generally, a character can run for a minute or two 
before having to rest for a minute

\vspace{12pt}
OVERLAND MOVEMENT

Characters covering long distances cross-country use overland movement. Overland 
movement is measured in miles per hour or miles per day. A day represents 8 hours 
of actual travel time. For rowed watercraft, a day represents 10 hours of rowing. 
For a sailing ship, it represents 24 hours.

\textbf{Walk:} A character can walk 8 hours in a day of travel without a problem. 
Walking for longer than that can wear him or her out (see Forced March, below).

\textbf{Hustle:} A character can hustle for 1 hour without a problem. Hustling 
for a second hour in between sleep cycles deals 1 point of nonlethal damage, and 
each additional hour deals twice the damage taken during the previous hour of hustling. 
A character who takes any nonlethal damage from hustling becomes fatigued.

A fatigued character can't run or charge and takes a penalty of -2 to Strength 
and Dexterity. Eliminating the nonlethal damage also eliminates the fatigue.

\textbf{Run:} A character can't run for an extended period of time.

Attempts to run and rest in cycles effectively work out to a hustle.

\textbf{Terrain:} The terrain through which a character travels affects how much 
distance he or she can cover in an hour or a day (see Table: Terrain and Overland 
Movement). A highway is a straight, major, paved road. A road is typically a dirt 
track. A trail is like a road, except that it allows only single-file travel and 
does not benefit a party traveling with vehicles. Trackless terrain is a wild area 
with no paths.

\textbf{Forced March:} In a day of normal walking, a character walks for 8 hours. 
The rest of the daylight time is spent making and breaking camp, resting, and eating.

A character can walk for more than 8 hours in a day by making a forced march. For 
each hour of marching beyond 8 hours, a Constitution check (DC 10, +2 per extra 
hour) is required. If the check fails, the character takes 1d6 points of nonlethal 
damage. A character who takes any nonlethal damage from a forced march becomes 
fatigued. Eliminating the nonlethal damage also eliminates the fatigue. It's possible 
for a character to march into unconsciousness by pushing himself too hard.

\textbf{Mounted Movement:} A mount bearing a rider can move at a hustle. The damage 
it takes when doing so, however, is lethal damage, not nonlethal damage. The creature 
can also be ridden in a forced march, but its Constitution checks automatically 
fail, and, again, the damage it takes is lethal damage. Mounts also become fatigued 
when they take any damage from hustling or forced marches.

See Table: Mounts and Vehicles for mounted speeds and speeds for vehicles pulled 
by draft animals.

\textbf{Waterborne Movement:} See Table: Mounts and Vehicles for speeds for water 
vehicles.

\vspace{12pt}
\begin{tabular}{|>{\raggedright}p{99pt}|>{\raggedright}p{48pt}|>{\raggedright}p{37pt}|>{\raggedright}p{37pt}|>{\raggedright}p{37pt}|}
\hline
\multicolumn{5}{|p{260pt}|}{T\textbf{able: Movement and Distance}}\tabularnewline
\hline
--------------------- & \multicolumn{4}{p{161pt}|}{ \textbf{Speed -------------------}}\tabularnewline
\hline
 & 15 feet & 20 feet & 30 feet & 40 feet\tabularnewline
\hline
O\textbf{ne Round (Tactical)}\textsuperscript{\textbf{1}} &  &  &  & \tabularnewline
\hline
Walk & 15 ft. & 20 ft. & 30 ft. & 40 ft.\tabularnewline
\hline
Hustle & 30 ft. & 40 ft. & 60 ft. & 80 ft.\tabularnewline
\hline
Run (x3) & 45 ft. & 60 ft. & 90 ft. & 120 ft.\tabularnewline
\hline
Run (x4) & 60 ft. & 80 ft. & 120 ft. & 160 ft.\tabularnewline
\hline
O\textbf{ne Minute (Local)} &  &  &  & \tabularnewline
\hline
Walk & 150 ft. & 200 ft. & 300 ft. & 400 ft.\tabularnewline
\hline
Hustle & 300 ft. & 400 ft. & 600 ft. & 800 ft.\tabularnewline
\hline
Run (x3) & 450 ft. & 600 ft. & 900 ft. & 1,200 ft.\tabularnewline
\hline
Run (x4) & 600 ft. & 800 ft. & 1,200 ft. & 1,600 ft.\tabularnewline
\hline
O\textbf{ne Hour (Overland)} &  &  &  & \tabularnewline
\hline
Walk & 1-1/2 miles & 2 miles & 3 miles & 4 miles\tabularnewline
\hline
Hustle & 3 miles & 4 miles & 6 miles & 8 miles\tabularnewline
\hline
Run--- & --- & --- & --- & \tabularnewline
\hline
O\textbf{ne Day (Overland)} &  &  &  & \tabularnewline
\hline
Walk & 12 miles & 16 miles & 24 miles & 32 miles\tabularnewline
\hline
Hustle--- & --- & --- & --- & \tabularnewline
\hline
Run--- & --- & --- & --- & \tabularnewline
\hline
\multicolumn{5}{|p{260pt}|}{1 Tactical movement is often measured in squares on 
the battle grid (1 square = 5 feet) rather than feet.}\tabularnewline
\hline
\end{tabular}

\vspace{12pt}
\begin{tabular}{|>{\raggedright}p{66pt}|>{\raggedright}p{118pt}|}
\hline
\multicolumn{2}{|p{185pt}|}{T\textbf{able: Hampered Movement}}\tabularnewline
\hline
C\textbf{ondition} & A\textbf{dditional Movement Cost}\tabularnewline
\hline
Difficult terrain & x2\tabularnewline
\hline
Obstacle\textsuperscript{1} & x2\tabularnewline
\hline
Poor visibility & x2\tabularnewline
\hline
Impassable--- & \tabularnewline
\hline
\multicolumn{2}{|p{185pt}|}{1 May require a skill check}\tabularnewline
\hline
\end{tabular}

\vspace{12pt}
\begin{tabular}{|>{\raggedright}p{62pt}|>{\raggedright}p{41pt}|>{\raggedright}p{62pt}|>{\raggedright}p{44pt}|}
\hline
\multicolumn{4}{|p{211pt}|}{\subsubsection*{T\textbf{able: Terrain and Overland 
Movement}}}\tabularnewline
\hline
T\textbf{errain } & H\textbf{ighway} & R\textbf{oad or Trail} & T\textbf{rackless}\tabularnewline
\hline
Desert, sandy & x1 & x1/2 & x1/2\tabularnewline
\hline
Forest & x1 & x1 & x1/2\tabularnewline
\hline
Hills & x1 & x3/4 & x1/2\tabularnewline
\hline
Jungle & x1 & x3/4 & x1/4\tabularnewline
\hline
Moor & x1 & x1 & x3/4\tabularnewline
\hline
Mountains & x3/4 & x3/4 & x1/2\tabularnewline
\hline
Plains & x1 & x1 & x3/4\tabularnewline
\hline
Swamp & x1 & x3/4 & x1/2\tabularnewline
\hline
Tundra, frozen & x1 & x3/4 & x3/4\tabularnewline
\hline
\end{tabular}

\vspace{12pt}
\begin{tabular}{|>{\raggedright}p{138pt}|>{\raggedright}p{63pt}|>{\raggedright}p{63pt}|}
\hline
\multicolumn{3}{|p{266pt}|}{T\textbf{able: Mounts and Vehicles}}\tabularnewline
\hline
M\textbf{ount/Vehicle} & P\textbf{er Hour} & P\textbf{er Day}\tabularnewline
\hline
Mount (carrying load) &  & \tabularnewline
\hline
Light horse or light warhorse & 6 miles & 48 miles\tabularnewline
\hline
Light horse (151-450 lb.)\textsuperscript{\textbf{1}} & 4 miles & 32 miles\tabularnewline
\hline
Light warhorse (231-690 lb.)\textsuperscript{\textbf{1}} & 4 miles & 32 miles\tabularnewline
\hline
Heavy horse or heavy warhorse & 5 miles & 40 miles\tabularnewline
\hline
Heavy horse (201-600 lb.)\textsuperscript{\textbf{1}} & 3-1/2 miles & 28 miles\tabularnewline
\hline
Heavy warhorse (301-900 lb.)\textsuperscript{\textbf{1}} & 3-1/2 miles & 28 miles\tabularnewline
\hline
Pony or warpony & 4 miles & 32 miles\tabularnewline
\hline
Pony (76-225 lb.)\textsuperscript{\textbf{1}} & 3 miles & 24 miles\tabularnewline
\hline
Warpony (101-300 lb.)\textsuperscript{\textbf{1}} & 3 miles & 24 miles\tabularnewline
\hline
Donkey or mule & 3 miles & 24 miles\tabularnewline
\hline
Donkey (51-150 lb.)\textsuperscript{\textbf{1}} & 2 miles & 16 miles\tabularnewline
\hline
Mule (231-690 lb.)\textsuperscript{\textbf{1}} & 2 miles & 16 miles\tabularnewline
\hline
Dog, riding & 4 miles & 32 miles\tabularnewline
\hline
Dog, riding (101-300 lb.)\textsuperscript{\textbf{1}} & 3 miles & 24 miles\tabularnewline
\hline
Cart or wagon & 2 miles & 16 miles\tabularnewline
\hline
Ship &  & \tabularnewline
\hline
Raft or barge (poled or towed)\textsuperscript{\textbf{2}} & 1/2 mile & 5 miles\tabularnewline
\hline
Keelboat (rowed)\textsuperscript{\textbf{2}} & 1 mile & 10 miles\tabularnewline
\hline
Rowboat (rowed)\textsuperscript{\textbf{2}} & 1-1/2 miles & 15 miles\tabularnewline
\hline
Sailing ship (sailed) & 2 miles & 48 miles\tabularnewline
\hline
Warship (sailed and rowed) & 2-1/2 miles & 60 miles\tabularnewline
\hline
Longship (sailed and rowed) & 3 miles & 72 miles\tabularnewline
\hline
Galley (rowed and sailed) & 4 miles & 96 miles\tabularnewline
\hline
\multicolumn{3}{|p{266pt}|}{1 Quadrupeds, such as horses, can carry heavier loads 
than characters can. See Carrying Capacity, above, for more information.}\tabularnewline
\hline
\multicolumn{3}{|p{266pt}|}{2 Rafts, barges, keelboats, and rowboats are used on 
lakes and rivers.\linebreak{}
If going downstream, add the speed of the current (typically 3 miles per hour) 
to the speed of the vehicle. In addition to 10 hours of being rowed, the vehicle 
can also float an additional 14 hours, if someone can guide it, so add an additional 
42 miles to the daily distance traveled. These vehicles can't be rowed against 
any significant current, but they can be pulled upstream by draft animals on the 
shores.}\tabularnewline
\hline
\end{tabular}

\vspace{12pt}
{\large{}MOVING IN THREE DIMENSIONS}

\vspace{12pt}
Tactical Aerial Movement

Once movement becomes three-dimensional and involves turning in midair and maintaining 
a minimum velocity to stay aloft, it gets more complicated. Most flying creatures 
have to slow down at least a little to make a turn, and many are limited to fairly 
wide turns and must maintain a minimum forward speed. Each flying creature has 
a maneuverability, as shown on Table: Maneuverability. The entries on the table 
are defined below.

\textit{Minimum Forward Speed: }If a flying creature fails to maintain its minimum 
forward speed, it must land at the end of its movement. If it is too high above 
the ground to land, it falls straight down, descending 150 feet in the first round 
of falling. If this distance brings it to the ground, it takes falling damage. 
If the fall doesn't bring the creature to the ground, it must spend its next turn 
recovering from the stall. It must succeed on a DC 20 Reflex save to recover. Otherwise 
it falls another 300 feet. If it hits the ground, it takes falling damage. Otherwise, 
it has another chance to recover on its next turn.

\textit{Hover: }The ability to stay in one place while airborne. 

\textit{Move Backward: }The ability to move backward without turning around.

\textit{Reverse: }A creature with good maneuverability uses up 5 feet of its speed 
to start flying backward.

\textit{Turn: }How much the creature can turn after covering the stated distance.

\textit{Turn in Place: }A creature with good or average maneuverability can use 
some of its speed to turn in place.

\textit{Maximum Turn: }How much the creature can turn in any one space. 

\textit{Up Angle: }The angle at which the creature can climb.

\textit{Up Speed: }How fast the creature can climb.

\textit{Down Angle: }The angle at which the creature can descend.

\textit{Down Speed: }A flying creature can fly down at twice its normal flying 
speed.

\textit{Between Down and Up: }An average, poor, or clumsy flier must fly level 
for a minimum distance after descending and before climbing. Any flier can begin 
descending after a climb without an intervening distance of level flight.

\vspace{12pt}
\begin{tabular}{|>{\raggedright}p{93pt}|>{\raggedright}p{31pt}|>{\raggedright}p{42pt}|>{\raggedright}p{42pt}|>{\raggedright}p{32pt}|>{\raggedright}p{36pt}|}
\hline
\multicolumn{6}{|p{278pt}|}{T\textbf{able: Maneuverability}}\tabularnewline
\hline
  & \multicolumn{5}{p{184pt}|}{M\textbf{aneuverability}}\tabularnewline
\hline
  & P\textbf{erfect} & G\textbf{ood} & A\textbf{verage} & P\textbf{oor} & C\textbf{lumsy}\tabularnewline
\hline
Minimum forward speed & None & None & Half & Half & Half\tabularnewline
\hline
Hover & Yes & Yes & No & No & No\tabularnewline
\hline
Move backward & Yes & Yes & No & No & No\tabularnewline
\hline
Reverse & Free- & 5 ft. & No & No & No\tabularnewline
\hline
Turn & Any & 90º/5 ft. & 45º/5 ft. & 45º/5 ft. & 45º/10 ft.\tabularnewline
\hline
Turn in place & Any & +90º/-5 ft. & +45º/-5 ft. & No & No\tabularnewline
\hline
Maximum turn & Any & Any & 90º & 45º & 45º\tabularnewline
\hline
Up angle & Any & Any & 60º & 45º & 45º\tabularnewline
\hline
Up speed & Full & Half & Half & Half & Half\tabularnewline
\hline
Down angle & Any & Any & Any & 45º & 45º\tabularnewline
\hline
Down speed & Double & Double & Double & Double & Double\tabularnewline
\hline
Between down and up & 0 & 0 & 5 ft. & 10 ft. & 20 ft.\tabularnewline
\hline
\end{tabular}

\vspace{12pt}
{\large{}EVASION AND PURSUIT}

In round-by-round movement, simply counting off squares, it's impossible for a 
slow character to get away from a determined fast character without mitigating 
circumstances. Likewise, it's no problem for a fast character to get away from 
a slower one. 

When the speeds of the two concerned characters are equal, there's a simple way 
to resolve a chase: If one creature is pursuing another, both are moving at the 
same speed, and the chase continues for at least a few rounds, have them make opposed 
Dexterity checks to see who is the faster over those rounds. If the creature being 
chased wins, it escapes. If the pursuer wins, it catches the fleeing creature. 

Sometimes a chase occurs overland and could last all day, with the two sides only 
occasionally getting glimpses of each other at a distance. In the case of a long 
chase, an opposed Constitution check made by all parties determines which can keep 
pace the longest. If the creature being chased rolls the highest, it gets away. 
If not, the chaser runs down its prey, outlasting it with stamina.

\vspace{12pt}
{\large{}MOVING AROUND IN SQUARES}

In general, when the characters aren't engaged in round-by-round combat, they should 
be able to move anywhere and in any manner that you can imagine real people could. 
A 5-foot square, for instance, can hold several characters; they just can't all 
fight effectively in that small space. The rules for movement are important for 
combat, but outside combat they can impose unnecessary hindrances on character 
activities.

\vspace{12pt}
{\LARGE{}EXPLORATION}

\vspace{12pt}
VISION AND LIGHT

Dwarves and half-orcs have darkvision, but everyone else needs light to see by. 
See Table: Light Sources and Illumination for the radius that a light source illuminates 
and how long it lasts.

In an area of bright light, all characters can see clearly. A creature can't hide 
in an area of bright light unless it is invisible or has cover.

In an area of shadowy illumination, a character can see dimly. Creatures within 
this area have concealment relative to that character. A creature in an area of 
shadowy illumination can make a Hide check to conceal itself.

In areas of darkness, creatures without darkvision are effectively blinded. In 
addition to the obvious effects, a blinded creature has a 50\% miss chance in combat 
(all opponents have total concealment), loses any Dexterity bonus to AC, takes 
a -2 penalty to AC, moves at half speed, and takes a -4 penalty on Search checks 
and most Strength and Dexterity-based skill checks.

Characters with low-light vision (elves, gnomes, and half-elves) can see objects 
twice as far away as the given radius. Double the effective radius of bright light 
and of shadowy illumination for such characters.

Characters with darkvision (dwarves and half-orcs) can see lit areas normally as 
well as dark areas within 60 feet. A creature can't hide within 60 feet of a character 
with darkvision unless it is invisible or has cover.

\vspace{12pt}
\begin{tabular}{|>{\raggedright}p{112pt}|>{\raggedright}p{53pt}|>{\raggedright}p{51pt}|>{\raggedright}p{46pt}|}
\hline
\multicolumn{4}{|p{264pt}|}{\subsubsection*{T\textbf{able: Light Sources and Illumination}}}\tabularnewline
\hline
O\textbf{bject} & B\textbf{right} & S\textbf{hadowy} & D\textbf{uration}\tabularnewline
\hline
Candle & n/a\textsuperscript{\textbf{1}} & 5 ft. & 1 hr.\tabularnewline
\hline
Everburning torch & 20 ft. & 40 ft. & Permanent\tabularnewline
\hline
Lamp, common & 15 ft. & 30 ft. & 6 hr./pint\tabularnewline
\hline
Lantern, bullseye\textsuperscript{\textbf{2}} & 60-ft. cone & 120-ft. cone & 6 
hr./pint\tabularnewline
\hline
Lantern, hooded & 30 ft. & 60 ft. & 6 hr./pint\tabularnewline
\hline
Sunrod & 30 ft. & 60 ft. & 6 hr.\tabularnewline
\hline
Torch & 20 ft. & 40 ft. & 1 hr.\tabularnewline
\hline
S\textbf{pell} & B\textbf{right} & S\textbf{hadowy} & D\textbf{uration}\tabularnewline
\hline
C\textit{ontinual flame} & 20 ft. & 40 ft. & Permanent\tabularnewline
\hline
D\textit{ancing lights }(torches) & 20 ft. (each) & 40 ft. (each) & 1 min.\tabularnewline
\hline
D\textit{aylight} & 60 ft. & 120 ft. & 30 min.\tabularnewline
\hline
L\textit{ight} & 20 ft. & 40 ft. & 10 min.\tabularnewline
\hline
\multicolumn{4}{|p{264pt}|}{1 A candle does not provide bright illumination, only 
shadowy illumination.}\tabularnewline
\hline
\multicolumn{4}{|p{264pt}|}{2 A bullseye lantern illuminates a cone, not a radius.}\tabularnewline
\hline
\end{tabular}

\vspace{12pt}
BREAKING AND ENTERING

When attempting to break an object, you have two choices: smash it with a weapon 
or break it with sheer strength.

\vspace{12pt}
\textbf{Smashing an Object}

Smashing a weapon or shield with a slashing or bludgeoning weapon is accomplished 
by the sunder special attack. Smashing an object is a lot like sundering a weapon 
or shield, except that your attack roll is opposed by the object's AC. Generally, 
you can smash an object only with a bludgeoning or slashing weapon.

\textbf{Armor Class:} Objects are easier to hit than creatures because they usually 
don't move, but many are tough enough to shrug off some damage from each blow. 
An object's Armor Class is equal to 10 + its size modifier + its Dexterity modifier. 
An inanimate object has not only a Dexterity of 0 (-5 penalty to AC), but also 
an additional -2 penalty to its AC. Furthermore, if you take a full-round action 
to line up a shot, you get an automatic hit with a melee weapon and a +5 bonus 
on attack rolls with a ranged weapon.

\textbf{Hardness:} Each object has hardness---a number that represents how well 
it resists damage. Whenever an object takes damage, subtract its hardness from 
the damage. Only damage in excess of its hardness is deducted from the object's 
hit points (see Table: Common Armor, Weapon, and Shield Hardness and Hit Points; 
Table: Substance Hardness and Hit Points; and Table: Object Hardness and Hit Points).

\textbf{Hit Points:} An object's hit point total depends on what it is made of 
and how big it is (see Table: Common Armor, Weapon, and Shield Hardness and Hit 
Points; Table: Substance Hardness and Hit Points; and Table: Object Hardness and 
Hit Points). When an object's hit points reach 0, it's ruined.

Very large objects have separate hit point totals for different sections.

\textit{Energy Attacks: }Acid and sonic attacks deal damage to most objects just 
as they do to creatures; roll damage and apply it normally after a successful hit. 
Electricity and fire attacks deal half damage to most objects; divide the damage 
dealt by 2 before applying the hardness. Cold attacks deal one-quarter damage to 
most objects; divide the damage dealt by 4 before applying the hardness.

\textit{Ranged Weapon Damage: }Objects take half damage from ranged weapons (unless 
the weapon is a siege engine or something similar). Divide the damage dealt by 
2 before applying the object's hardness.

\textit{Ineffective Weapons: }Certain weapons just can't effectively deal damage 
to certain objects.

\textit{Immunities}: Objects are immune to nonlethal damage and to critical hits.

Even animated objects, which are otherwise considered creatures, have these immunities 
because they are constructs.

\textit{Magic Armor, Shields, and Weapons: }Each +1 of enhancement bonus adds 2 
to the hardness of armor, a weapon, or a shield and +10 to the item's hit points.

\textit{Vulnerability to Certain Attacks: }Certain attacks are especially successful 
against some objects. In such cases, attacks deal double their normal damage and 
may ignore the object's hardness.

\textit{Damaged Objects: }A damaged object remains fully functional until the item's 
hit points are reduced to 0, at which point it is destroyed.

Damaged (but not destroyed) objects can be repaired with the Craft skill.

\textbf{Saving Throws: }Nonmagical, unattended items never make saving throws. 
They are considered to have failed their saving throws, so they always are affected 
by spells. An item attended by a character (being grasped, touched, or worn) makes 
saving throws as the character (that is, using the character's saving throw bonus).

Magic items always get saving throws. A magic item's Fortitude, Reflex, and Will 
save bonuses are equal to 2 + one-half its caster level. An attended magic item 
either makes saving throws as its owner or uses its own saving throw bonus, whichever 
is better.

\textit{Animated Objects: }Animated objects count as creatures for purposes of 
determining their Armor Class (do not treat them as inanimate objects).

\vspace{12pt}
\textbf{Breaking Items}

When a character tries to break something with sudden force rather than by dealing 
damage, use a Strength check (rather than an attack roll and damage roll, as with 
the sunder special attack) to see whether he or she succeeds. The DC depends more 
on the construction of the item than on the material.

If an item has lost half or more of its hit points, the DC to break it drops by 
2.

Larger and smaller creatures get size bonuses and size penalties on Strength checks 
to break open doors as follows: Fine -16, Diminutive -12, Tiny -8, Small -4, Large 
+4, Huge +8, Gargantuan +12, Colossal +16.

A crowbar or portable ram improves a character's chance of breaking open a door.

\vspace{12pt}
\begin{tabular}{|>{\raggedright}p{164pt}|>{\raggedright}p{42pt}|>{\raggedright}p{69pt}|}
\hline
\multicolumn{3}{|p{275pt}|}{T\textbf{able: Common Armor, Weapon, and Shield Hardness 
and Hit Points}}\tabularnewline
\hline
W\textbf{eapon or Shield} & H\textbf{ardness} & H\textbf{P}\textsuperscript{\textbf{1}}\tabularnewline
\hline
Light blade & 10 & 2\tabularnewline
\hline
One-handed blade & 10 & 5\tabularnewline
\hline
Two-handed blade & 10 & 10\tabularnewline
\hline
Light metal-hafted weapon & 10 & 10\tabularnewline
\hline
One-handed metal-hafted weapon & 10 & 20\tabularnewline
\hline
Light hafted weapon & 5 & 2\tabularnewline
\hline
One-handed hafted weapon & 5 & 5\tabularnewline
\hline
Two-handed hafted weapon & 5 & 10\tabularnewline
\hline
Projectile weapon & 5 & 5\tabularnewline
\hline
Armor & special\textsuperscript{\textbf{2}} & armor bonus x$ $5\tabularnewline
\hline
Buckler & 10 & 5\tabularnewline
\hline
Light wooden shield & 5 & 7\tabularnewline
\hline
Heavy wooden shield & 5 & 15\tabularnewline
\hline
Light steel shield & 10 & 10\tabularnewline
\hline
Heavy steel shield & 10 & 20\tabularnewline
\hline
Tower shield & 5 & 20\tabularnewline
\hline
\multicolumn{3}{|p{275pt}|}{1 The hp value given is for Medium armor, weapons, 
and shields.\linebreak{}
Divide by 2 for each size category of the item smaller than Medium, or multiply 
it by 2 for each size category larger than Medium.}\tabularnewline
\hline
\multicolumn{3}{|p{275pt}|}{2 Varies by material; see Table: Substance Hardness 
and Hit Points.}\tabularnewline
\hline
\end{tabular}

\vspace{12pt}
\begin{tabular}{|>{\raggedright}p{59pt}|>{\raggedright}p{39pt}|>{\raggedright}p{79pt}|}
\hline
\multicolumn{3}{|p{178pt}|}{T\textbf{able: Substance Hardness and Hit Points}}\tabularnewline
\hline
S\textbf{ubstance} & H\textbf{ardness} & H\textbf{it Points}\tabularnewline
\hline
Paper or cloth & 0 & 2/inch of thickness\tabularnewline
\hline
Rope & 0 & 2/inch of thickness\tabularnewline
\hline
Glass & 1 & 1/inch of thickness\tabularnewline
\hline
Ice & 0 & 3/inch of thickness\tabularnewline
\hline
Leather or hide & 2 & 5/inch of thickness\tabularnewline
\hline
Wood & 5 & 10/inch of thickness\tabularnewline
\hline
Stone & 8 & 15/inch of thickness\tabularnewline
\hline
Iron or steel & 10 & 30/inch of thickness\tabularnewline
\hline
Mithral & 15 & 30/inch of thickness\tabularnewline
\hline
Adamantine & 20 & 40/inch of thickness\tabularnewline
\hline
\end{tabular}

\vspace{12pt}
\begin{tabular}{|>{\raggedright}p{78pt}|>{\raggedright}p{78pt}|}
\hline
\multicolumn{2}{|p{156pt}|}{T\textbf{able: Size and Armor Class of Objects}}\tabularnewline
\hline
S\textbf{ize} & A\textbf{C Modifier}\tabularnewline
\hline
Colossal- & 8\tabularnewline
\hline
Gargantuan- & 4\tabularnewline
\hline
Huge- & 2\tabularnewline
\hline
Large- & 1\tabularnewline
\hline
Medium & +0\tabularnewline
\hline
Small & +1\tabularnewline
\hline
Tiny & +2\tabularnewline
\hline
Diminutive & +4\tabularnewline
\hline
Fine & +8\tabularnewline
\hline
\end{tabular}

\vspace{12pt}
\begin{tabular}{|>{\raggedright}p{101pt}|>{\raggedright}p{39pt}|>{\raggedright}p{41pt}|>{\raggedright}p{41pt}|}
\hline
\multicolumn{4}{|p{225pt}|}{T\textbf{able: Object Hardness and Hit Points}}\tabularnewline
\hline
O\textbf{bject} & H\textbf{ardness} & H\textbf{it Points} & B\textbf{reak DC}\tabularnewline
\hline
Rope (1 inch diam.) & 0 & 2 & 23\tabularnewline
\hline
Simple wooden door & 5 & 10 & 13\tabularnewline
\hline
Small chest & 5 & 1 & 17\tabularnewline
\hline
Good wooden door & 5 & 15 & 18\tabularnewline
\hline
Treasure chest & 5 & 15 & 23\tabularnewline
\hline
Strong wooden door & 5 & 20 & 23\tabularnewline
\hline
Masonry wall (1 ft. thick) & 8 & 90 & 35\tabularnewline
\hline
Hewn stone (3 ft. thick) & 8 & 540 & 50\tabularnewline
\hline
Chain & 10 & 5 & 26\tabularnewline
\hline
Manacles & 10 & 10 & 26\tabularnewline
\hline
Masterwork manacles & 10 & 10 & 28\tabularnewline
\hline
Iron door (2 in. thick) & 10 & 60 & 28\tabularnewline
\hline
\end{tabular}

\vspace{12pt}
\begin{tabular}{|>{\raggedright}p{97pt}|>{\raggedright}p{69pt}|}
\hline
\multicolumn{2}{|p{166pt}|}{T\textbf{able: DCs to Break or Burst Items}}\tabularnewline
\hline
S\textbf{trength Check to:} & D\textbf{C}\tabularnewline
\hline
Break down simple door & 13\tabularnewline
\hline
Break down good door & 18\tabularnewline
\hline
Break down strong door & 23\tabularnewline
\hline
Burst rope bonds & 23\tabularnewline
\hline
Bend iron bars & 24\tabularnewline
\hline
Break down barred door & 25\tabularnewline
\hline
Burst chain bonds & 26\tabularnewline
\hline
Break down iron door & 28\tabularnewline
\hline
C\textbf{ondition} & D\textbf{C Adjustment}\textsuperscript{\textbf{1}}\tabularnewline
\hline
H\textit{old portal} & +5\tabularnewline
\hline
A\textit{rcane lock} & +10\tabularnewline
\hline
\multicolumn{2}{|p{166pt}|}{1 If both apply, use the larger number.}\tabularnewline
\hline
\end{tabular}

\newpage

\end{document}
